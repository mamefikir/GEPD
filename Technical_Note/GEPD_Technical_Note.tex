\documentclass[]{article}
\usepackage{lmodern}
\usepackage{amssymb,amsmath}
\usepackage{ifxetex,ifluatex}
\usepackage{fixltx2e} % provides \textsubscript
\ifnum 0\ifxetex 1\fi\ifluatex 1\fi=0 % if pdftex
  \usepackage[T1]{fontenc}
  \usepackage[utf8]{inputenc}
\else % if luatex or xelatex
  \ifxetex
    \usepackage{mathspec}
  \else
    \usepackage{fontspec}
  \fi
  \defaultfontfeatures{Ligatures=TeX,Scale=MatchLowercase}
\fi
% use upquote if available, for straight quotes in verbatim environments
\IfFileExists{upquote.sty}{\usepackage{upquote}}{}
% use microtype if available
\IfFileExists{microtype.sty}{%
\usepackage{microtype}
\UseMicrotypeSet[protrusion]{basicmath} % disable protrusion for tt fonts
}{}
\usepackage[margin=1in]{geometry}
\usepackage{hyperref}
\hypersetup{unicode=true,
            pdftitle={GEPD Technical Note},
            pdfauthor={Brian Stacy},
            pdfborder={0 0 0},
            breaklinks=true}
\urlstyle{same}  % don't use monospace font for urls
\usepackage{color}
\usepackage{fancyvrb}
\newcommand{\VerbBar}{|}
\newcommand{\VERB}{\Verb[commandchars=\\\{\}]}
\DefineVerbatimEnvironment{Highlighting}{Verbatim}{commandchars=\\\{\}}
% Add ',fontsize=\small' for more characters per line
\usepackage{framed}
\definecolor{shadecolor}{RGB}{248,248,248}
\newenvironment{Shaded}{\begin{snugshade}}{\end{snugshade}}
\newcommand{\AlertTok}[1]{\textcolor[rgb]{0.94,0.16,0.16}{#1}}
\newcommand{\AnnotationTok}[1]{\textcolor[rgb]{0.56,0.35,0.01}{\textbf{\textit{#1}}}}
\newcommand{\AttributeTok}[1]{\textcolor[rgb]{0.77,0.63,0.00}{#1}}
\newcommand{\BaseNTok}[1]{\textcolor[rgb]{0.00,0.00,0.81}{#1}}
\newcommand{\BuiltInTok}[1]{#1}
\newcommand{\CharTok}[1]{\textcolor[rgb]{0.31,0.60,0.02}{#1}}
\newcommand{\CommentTok}[1]{\textcolor[rgb]{0.56,0.35,0.01}{\textit{#1}}}
\newcommand{\CommentVarTok}[1]{\textcolor[rgb]{0.56,0.35,0.01}{\textbf{\textit{#1}}}}
\newcommand{\ConstantTok}[1]{\textcolor[rgb]{0.00,0.00,0.00}{#1}}
\newcommand{\ControlFlowTok}[1]{\textcolor[rgb]{0.13,0.29,0.53}{\textbf{#1}}}
\newcommand{\DataTypeTok}[1]{\textcolor[rgb]{0.13,0.29,0.53}{#1}}
\newcommand{\DecValTok}[1]{\textcolor[rgb]{0.00,0.00,0.81}{#1}}
\newcommand{\DocumentationTok}[1]{\textcolor[rgb]{0.56,0.35,0.01}{\textbf{\textit{#1}}}}
\newcommand{\ErrorTok}[1]{\textcolor[rgb]{0.64,0.00,0.00}{\textbf{#1}}}
\newcommand{\ExtensionTok}[1]{#1}
\newcommand{\FloatTok}[1]{\textcolor[rgb]{0.00,0.00,0.81}{#1}}
\newcommand{\FunctionTok}[1]{\textcolor[rgb]{0.00,0.00,0.00}{#1}}
\newcommand{\ImportTok}[1]{#1}
\newcommand{\InformationTok}[1]{\textcolor[rgb]{0.56,0.35,0.01}{\textbf{\textit{#1}}}}
\newcommand{\KeywordTok}[1]{\textcolor[rgb]{0.13,0.29,0.53}{\textbf{#1}}}
\newcommand{\NormalTok}[1]{#1}
\newcommand{\OperatorTok}[1]{\textcolor[rgb]{0.81,0.36,0.00}{\textbf{#1}}}
\newcommand{\OtherTok}[1]{\textcolor[rgb]{0.56,0.35,0.01}{#1}}
\newcommand{\PreprocessorTok}[1]{\textcolor[rgb]{0.56,0.35,0.01}{\textit{#1}}}
\newcommand{\RegionMarkerTok}[1]{#1}
\newcommand{\SpecialCharTok}[1]{\textcolor[rgb]{0.00,0.00,0.00}{#1}}
\newcommand{\SpecialStringTok}[1]{\textcolor[rgb]{0.31,0.60,0.02}{#1}}
\newcommand{\StringTok}[1]{\textcolor[rgb]{0.31,0.60,0.02}{#1}}
\newcommand{\VariableTok}[1]{\textcolor[rgb]{0.00,0.00,0.00}{#1}}
\newcommand{\VerbatimStringTok}[1]{\textcolor[rgb]{0.31,0.60,0.02}{#1}}
\newcommand{\WarningTok}[1]{\textcolor[rgb]{0.56,0.35,0.01}{\textbf{\textit{#1}}}}
\usepackage{longtable,booktabs}
\usepackage{graphicx,grffile}
\makeatletter
\def\maxwidth{\ifdim\Gin@nat@width>\linewidth\linewidth\else\Gin@nat@width\fi}
\def\maxheight{\ifdim\Gin@nat@height>\textheight\textheight\else\Gin@nat@height\fi}
\makeatother
% Scale images if necessary, so that they will not overflow the page
% margins by default, and it is still possible to overwrite the defaults
% using explicit options in \includegraphics[width, height, ...]{}
\setkeys{Gin}{width=\maxwidth,height=\maxheight,keepaspectratio}
\IfFileExists{parskip.sty}{%
\usepackage{parskip}
}{% else
\setlength{\parindent}{0pt}
\setlength{\parskip}{6pt plus 2pt minus 1pt}
}
\setlength{\emergencystretch}{3em}  % prevent overfull lines
\providecommand{\tightlist}{%
  \setlength{\itemsep}{0pt}\setlength{\parskip}{0pt}}
\setcounter{secnumdepth}{0}
% Redefines (sub)paragraphs to behave more like sections
\ifx\paragraph\undefined\else
\let\oldparagraph\paragraph
\renewcommand{\paragraph}[1]{\oldparagraph{#1}\mbox{}}
\fi
\ifx\subparagraph\undefined\else
\let\oldsubparagraph\subparagraph
\renewcommand{\subparagraph}[1]{\oldsubparagraph{#1}\mbox{}}
\fi

%%% Use protect on footnotes to avoid problems with footnotes in titles
\let\rmarkdownfootnote\footnote%
\def\footnote{\protect\rmarkdownfootnote}

%%% Change title format to be more compact
\usepackage{titling}

% Create subtitle command for use in maketitle
\providecommand{\subtitle}[1]{
  \posttitle{
    \begin{center}\large#1\end{center}
    }
}

\setlength{\droptitle}{-2em}

  \title{GEPD Technical Note}
    \pretitle{\vspace{\droptitle}\centering\huge}
  \posttitle{\par}
    \author{Brian Stacy}
    \preauthor{\centering\large\emph}
  \postauthor{\par}
      \predate{\centering\large\emph}
  \postdate{\par}
    \date{1/21/2020}


\begin{document}
\maketitle

{
\setcounter{tocdepth}{2}
\tableofcontents
}
\hypertarget{background-on-project}{%
\section{Background on Project}\label{background-on-project}}

Policymakers in low- and middle-income countries who are working to
improve student learning often find themselves flying blind. They see
the budget that goes into education and (sometimes) the learning that
students come out with, but they lack information on the crucial factors
in between--- the practices, policies, and politics---that drive those
learning outcomes. The new Global Education Policy Dashboard initiative
will fill that gap.

\hypertarget{the-challenge-understanding-the-learning-crisis}{%
\subsection{The Challenge: Understanding the Learning
Crisis}\label{the-challenge-understanding-the-learning-crisis}}

Many countries, despite having significantly increased access to
education for their children and youth, now realize that they are facing
a learning crisis. For instance, when grade 3 students in Nicaragua were
tested in 2011, only half correctly solved 5 + 6. And in Kenya,
Tanzania, and Uganda, when grade 3 students were asked recently to read
a sentence such as ``The name of the dog is Puppy,'' three-quarters
could not answer what the dog's name was. Examples like these from
around the world underline that schooling is not the same as
learning---even though education policy often assumes that it is.

The World Development Report 2018 argued that the learning crisis has
both micro- and system-level causes: poor service delivery in schools
and communities, driven by unhealthy politics and low bureaucratic
capacity, mediated through policies that are not aligned toward learning
for all. To tackle the crisis and improve student learning at scale,
countries need to know where they stand on all three of these
dimensions---practices (or service delivery), policies, and politics.

But providing such a systemwide overview requires better measurement.
Many of these drivers of learning are not captured by existing
administrative systems. And although new measurement tools capture some
of those aspects well, no single instrument pulls together data on all
these areas. This gap leaves policymakers in the dark about what is
working and what isn't.

\hypertarget{our-response-shining-a-light-on-the-drivers-of-learning}{%
\subsection{Our Response: Shining a light on the drivers of
learning}\label{our-response-shining-a-light-on-the-drivers-of-learning}}

To fill this gap, the World Bank, with support from the Bill and Melinda
Gates Foundation and the UK's Department for International Development,
is designing a Global Education Policy Dashboard, which will measure the
drivers of learning outcomes in basic education around the world. In
doing so, it will highlight gaps between current practice and what the
evidence suggests would be most effective in promoting learning, and it
will give governments a way to set priorities and track progress as they
work to close those gaps. This collaboration will advance the goals of
the Human Capital Project, a global effort to accelerate more and better
investments in people for greater equity and economic growth.

The dashboard will start by focusing attention on education
outcomes---learning and participation in primary school. The next set of
indicators will measure the quality of service delivery, focusing on
four key school-level ingredients of student learning: teaching, school
management, technology for learning, and learner preparation.

In addition, the dashboard will measure deeper systemic drivers: the
policies and politics that determine the quality of service delivery.
The end goal is to have a set of indicators that is comprehensive but
also focused so stakeholders can pay attention to what is really most
important.

The dashboard will be implemented in 13 countries in 2019, with a goal
of rapidly expanding coverage after lessons from the field have been
incorporated.

\hypertarget{our-approach-how-the-dashboard-will-achieve-this}{%
\subsection{Our Approach: How the Dashboard will achieve
this}\label{our-approach-how-the-dashboard-will-achieve-this}}

Develop tools to collect data at all levels of the system on a regular
basis. Because of the shortcomings of existing data on the drivers of
learning, in its early stage the dashboard will rely primarily on data
collected under this initiative, rather than just presenting existing
data in a new framework. In each participating country, the initiative
will deploy various new data-collection instruments, including a School
Survey, an Expert Survey, and a Survey of Public Officials. This
multilevel approach is necessary to gather data on all the drivers of
learning, from the level of the learner to the level of the system. The
aim is to develop lean and cost-effective instruments so the dashboard
can be inexpensive enough to be applied repeatedly and to be scaled up
after initial phasing.

Embed innovative measurement approaches in a coherent, system-wide
framework. The dashboard approach combines the conceptual framework of
the World Development Report 2018 with indicators derived from
streamlined versions of the most current measurement tools. These tools
include the Service Delivery Indicators (SDI) initiative, Systems
Approach for Better Results (SABER), Global Early Child Development
Database (GECDD), World Management Survey, Bureaucracy Lab surveys, and
others.

Present a user-friendly dashboard interface. After gathering data, the
dashboard initiative will display the indicators in an easy-to-use
interface, focusing attention on key constraints. The primary audience
will be policymakers, placing a premium on a visual representation that
illuminates key relationships and constraints on learning.

Make resources available for implementation in all developing countries.
Along with developing the dashboard and implementing it in an initial
set of 13 countries, the World Bank and partners will create tools and
resources to make it easy for countries to use it to strengthen their
own administrative systems. Ultimately, this will allow global scale-up
to all developing countries that find it useful.

\hypertarget{development-of-the-dashboard-instruments}{%
\section{Development of the Dashboard
Instruments}\label{development-of-the-dashboard-instruments}}

The dashboard project collects new data in each country using three new
instruments: A School Survey, a Policy Survey, and a Survey of Public
Officials. Data collection involves school visits, classroom
observations, legislative reviews, teacher and student assessments, and
interviews with teachers, principals, and public officials. In addition,
the project draws on some existing data sources to complement the new
data it collects. A major objective of the GEPD project was to develop
focused, cost-effective instruments and data-collection procedures, so
that the dashboard can be inexpensive enough to be applied (and
re-applied) in many countries. The team achieved this by streamlining
and simplifying existing instruments, and thereby reducing the time
required for data collection and training of enumerators.

More information pertaining to each of the three instruments can be
found in the Global Education POlicy Dashboard Booklet.

\hypertarget{proficiency-cutoffs}{%
\section{Proficiency Cutoffs}\label{proficiency-cutoffs}}

For our learning indicators and our indicator of teaching proficiency,
we had to set thresholds for minimal proficiency. Below we will discuss
how these proficiency cutoffs were determined.

\hypertarget{student-learning-assessment}{%
\subsubsection{Student Learning
Assessment}\label{student-learning-assessment}}

The SDI student test was designed as a one-on-one evaluation with
enumerators reading out instructions to students in their mother tongue.
This was done so as to build up a differentiated picture of students'
cognitive skills; i.e.~oral one-to-one testing which allows testing
whether a child can solve a mathematics problem even when his/her
reading ability is so low that he/she would not be able to attempt the
problem independently. The language test (implemented in the language of
instruction: English, French, Swahili, or Portuguese) ranged from simple
tasks testing letter and word recognition to a more challenging reading
comprehension test. The mathematics test ranged from recognizing and
ordering numbers, to addition of one- to three-digit numbers, to one-
and two-digit subtraction, to single digit multiplication and division.

The sub-domains for the literacy section are:

\begin{itemize}
\tightlist
\item
  Letter Identification
\item
  Word Recognition
\item
  Reading Comprehension Story I\\
\item
  Reading Comprehension Story II
\end{itemize}

The Reading Comprehension Story II set of items were added to the
standard SDI assessment by our team, and came from a publicly released
items from the PIRLS assessment.

The sub-domains for the math section are:

\begin{itemize}
\tightlist
\item
  Number Sense
\item
  Arithmetic
\item
  Word Problem
\item
  Sequences
\end{itemize}

In order to determine whether or not a child was proficient based on our
assessment, out team consulted with a set of experts to complete a
standard setting exercise where each expert was asked to rate whether a
minimally proficient student in the subject should be able to answer the
item correctly using the global proficiency framework being designed by
UIS. The result of that exercise was that we would expect a minimally
proficient child to answer 20/24 questions correctly on our literacy
section and 14/17 points on our mathematics assessment. Students scoring
at least this many points on our assessment are rated as proficient and
otherwise rated not proficient.

INCLUDE LIST OF EXPERTS WE CONSULTED TO THANK THEM HERE

\hypertarget{teacher-content-knowledge-assessment}{%
\subsubsection{Teacher Content Knowledge
Assessment}\label{teacher-content-knowledge-assessment}}

The objective of the teacher test is three-fold. The first objective is
to examine whether teachers have the mastery of the subjects they are
teaching. This is interpreted as the minimum knowledge required for the
teacher to be effective. In addition, the test also examines the extent
to which teachers demonstrate mastery of subject content skills that are
above the level they are teaching at and finally, mastery of pedagogic
skills.

To simplify the administration of the test it was designed as a marking
exercise, in which teachers are asked to mark and correct a hypothetical
student's exam. The Language test was administered to teachers teaching
language, or language and other subjects, and the mathematics test was
administered to teacher teaching mathematics, or mathematics and other
subjects. The test was validated against 13 Sub-Saharan primary
curricula.\footnote{The countries included for the review were:
  Botswana, Ethiopia, Gambia, Kenya, Magadascar, Mauritius, Namibia,
  Nigeria, Rwanda, Seychelles, South Africa, Tanzania and Uganda. See
  David Johnson, Andrew Cunningham and Rachel Dowling (2012) ``Teaching
  Standards and Curriculum Review'', prepared as background document for
  the SDI Survey.} \footnote{While some SDI countries were not part of
  the review, before doing the SDI survey the SDI team verified that the
  school curriculum was compatible with the SDI teacher and student
  test. For more information see SDI Technical report for each specific
  country.}

The language test consisted of two sections. The first section asked
teachers to assess pupil language literacy by correcting a primary
school pupil language test. The teachers have to correct whether the
``student answer'' is correct, and in case it is incorrect, has to write
the correct answer. The second section asks teachers to correct a letter
written by a child in 4th grade. The teachers have to correct the letter
for grammar, punctuation (between sentences and within sentences),
spelling, syntax, and salutation by circling the mistakes and writing
the correction on the line. The mathematics test asks teachers to assess
pupil numeracy literacy by correcting a primary school pupil mathematic
test. The teachers have to correct whether the ``student answer'' is
correct, and in case it is incorrect it has to write the correct answer.

The pedagogy test consists of three sections designed to capture all the
skills teachers would routinely be asked to apply when teaching. The
first section asks teachers to prepare a lesson plan about road
accidents based on a simple information given a text they had read. The
second task asks teachers to assess children's writing on the basis of
two sample letters (written by grade 4 children). Teachers have to point
out the strengths and weaknesses on the student's letters, by evaluating
pupil's ability to write simple letters, use sentence structures
correctly including past and present tenses, use a range of vocabulary,
and use both within and between sentence punctuation. The final task
asks teachers to inspect test scores of 10 children, aggregate them and
make some statements about learning patterns.

We set a threshold of 80\% on the math and language assessment for a
teacher being minimally proficient. This standard was used in the
Service Delivery Indicators initiative, and we are using it as a
benchmark as well.

\hypertarget{st-grade-assessment}{%
\subsubsection{1st Grade Assessment}\label{st-grade-assessment}}

Our first grade assessment tool was derived mostly from the MELQO
assessment developed as an open source tool to measure early childhood
development in the domains of numeracy, literacy, socio-emotional skills
and numeracy. For more details on the development of this initiative
see:

\url{https://unesdoc.unesco.org/ark:/48223/pf0000248053}

The MELQO instrument was designed to measure development of children
around the age of six years old. Our target group, 1st graders, is
slightly older and we did some adaptation in which we removed some items
and added items on simple sentence reading to better match what is
expected of older children.

The sub-domains for the literacy section are:

\begin{itemize}
\tightlist
\item
  Expressive Vocabulary
\item
  Letter Identification
\item
  Word Recognition
\item
  Sentence Reading
\item
  Listening Comprehension Story
\item
  Name Writing
\item
  Print Awareness
\end{itemize}

The sub-domains for the math section are:

\begin{itemize}
\tightlist
\item
  Verbal Counting
\item
  Producing a Set
\item
  Number Identification
\item
  Number Comparison
\item
  Simple Addition
\end{itemize}

The sub-domains for the socio-emotional section are:

\begin{itemize}
\tightlist
\item
  Perspective Taking/Empathy
\item
  Conflict Resolution
\end{itemize}

The sub-domains for the executive functioning section are:

\begin{itemize}
\tightlist
\item
  Working Memory/Backward Digit Spans
\item
  Follow Instructions/Head, Toes, Knees, Shoulders Task
\end{itemize}

Our threshold for proficiency on our 1st grade assessment is to score at
least 80\% of the items correctly. We validated this approach by
calculating the percent correct in the top 25\% highest scoring schools
in Peru on our 4th grade dashboard assessment and found that students in
these schools answered around this percentage of items correct.

\hypertarget{stop-light-scoring-system}{%
\section{Stop Light Scoring System}\label{stop-light-scoring-system}}

We use the following system for categorizing our main indicators into
three levels: On Target, Caution, and Needs Improvement.

For our indicators ranging from 0-100, we use the following system:

\begin{enumerate}
\def\labelenumi{\arabic{enumi}.}
\tightlist
\item
  On Target - Values at least 90\%\\
\item
  Caution - Values between 70-90\\
\item
  Needs improvement - Values under 70\%
\end{enumerate}

For our indicators ranging 1-5, we use the following system:

\begin{enumerate}
\def\labelenumi{\arabic{enumi}.}
\tightlist
\item
  On Target - Values at least 4\\
\item
  Caution - Values between 2 and 4\\
\item
  Needs improvement - Values 2 and under
\end{enumerate}

\hypertarget{sampling}{%
\section{Sampling}\label{sampling}}

\hypertarget{general-notes-on-sampling}{%
\subsection{General Notes on Sampling}\label{general-notes-on-sampling}}

The aim of the Global Education Policy Dashboard school survey is to
produce nationally representative estimates, which will be able to
detect changes in the indicators over time at a minimum power of 80\%
and with a 0.05 significance level. We also wish to detect differences
by urban/rural location.

For our school survey, we will employ a two-stage random sample design,
where in the first stage a sample of typically around 200 schools, based
on local conditions, is drawn, chosen in advance by the Bank staff. In
the second stage, a sample of teachers and students will be drawn to
answer questions from our survey modules, chosen in the field. A total
of 10 teachers will be sampled for absenteeism. Five teachers will be
interviewed and given a content knowledge exam. Three 1st grade students
will be assessed at random, and a classroom of 4th grade students will
be assessed at random. Stratification will be based on the school's
urban/rural classification and based on region. When stratifying by
region, we will work with our partners within the country to make sure
we include all relevant geographical divisions.

For our Survey of Public Officials, we will sample a total of 200 public
officials. Roughly 60 officials are typically surveyed at the federal
level, while 140 officials will be surveyed at the regional/district
level. For selection of officials at the regional and district level, we
will employ a cluster sampling strategy, where roughly 10 regional
offices (or whatever the secondary administrative unit is called) are
chosen at random from among the regions in which schools were sampled.
Then among these 10 regions, we also typically select around 10
districts (tertiary administrative level units) from among the districts
in which schools werer sampled. The result of this sampling approach is
that for 10 clusters we will have links from the school to the district
office office to the regional office to the central office. Within the
regions/districts, five or six officials will be sampled, including the
head of organization, HR director, two division directors from finance
and planning, and one or two randomly selected professional employees
among the finance, planning, and one other service related department
chosen at random. At the federal level, we will interview the HR
director, finance director, planning director, and three randomly
selected service focused departments. In addition to the directors of
each of these departments, a sample of 9 professional employees will be
chosen in each department at random on the day of the interview.

\hypertarget{detailed-sampling-notes}{%
\subsection{Detailed Sampling Notes}\label{detailed-sampling-notes}}

Our team consulted with several sampling experts at the World Bank to
discuss strategies for sampling schools. As mentioned, the goal of our
survey is to be able to detect differences across time and by
urban/rural location. The result of our consultation with experts was to
use an optimal stratification strategy based on the optimal stratified
sampling technique in Barcaroli (2014) which uses the genetic algorithm
to produce an optimal number of strata and an optimal number of units
within strata. Optimal stratification can lead to reduced sampling error
compared to alternative approaches such as cluster sampling or simple
random sampling.

Before producing the sample using the optimal stratification approach,
we would define the sampling frame. Typically, we would take all schools
with at least three 1st grade and at least three 4th grade students,
according to the latest school census, for inclusion in the frame. A
minimum of three 1st grade and 4th grade students was set to ensure that
a sufficient number of teachers in primary school and students would be
interviewed. In order to do this, we work with the World Bank country
teams and the country government to compile an up to date and detailed
database containing information on schools. Depending on the country, we
may or may not include private schools in the sampling frame depending
on the particular country context. The decision to include private
schools is usually determined by the share of private schools that make
up the total student enrollment and the ease with which we are able to
enter private schools in the country. We work with the government and
the local World Bank team in making this decision. In Peru, we included
only public schools in the sampling frame, because private schools make
up only a small percentage of total student enrollment.

In our stratified sampling approach, we typically use a countries 1st or
2nd level administative division (which usually is a province or
department) depending on the country and the urban/rural status of the
school as stratification variables. For example, in Peru, the
Departamento and urban/rural status of the school made up the
stratification variables. In cases, where private schools are also
included in the sampling frame, the public/private status of the school
is also included as a stratification variable.

The optimal stratification apporach is particularly useful if previous
surveys have been done in the country which collect information similar
to the information collected in our Dashboard. The optimal
stratification algorith will assign extra units to a stratum that has a
particularly variance in our target variables. For instance in Peru, we
used data from a previous nationally representative survey which used
the MELQO instrument to test Early Childhoold skills. We were able to
calculate the mean and standard deviation of the student level test
scores by departamento, and we were able to assign extra sampling units
to stratum with relatively high variances in these outcomes. To the
extent that regions with a high variance in the MELQO test also have a
high variance in our other outcomes (such as the 4th grade test scores,
teacher knowledge, principal practices), we would expect this sampling
approach to improve precision in these measures as well.

A quick introduction to the technical aspects of the opimal
stratification approach can be found here.

\url{https://cran.r-project.org/web/packages/SamplingStrata/vignettes/SamplingStrata.html}

In all cases, where the optimal stratification algorithm was applied, we
also in paralell ran a sampling approach based on a more simple
stratified sampling approach where units were broken into strata and the
number of units per strata was proportionate to the stratum size.
Typically, we did not find major differences in the summary statistics
of the sample (in terms of average numbers of students per school, or
number of schools per region) based on the optimal stratification
approach compared to a more simple stratification approach.

Barcaroli G (2014). ``SamplingStrata: An R Package for the Optimization
of Stratified Sampling.'' Journal of Statistical Software, 61(4), 1--24.
\url{http://www.jstatsoft.org/v61/i04/}.

\hypertarget{peru-specific-comments}{%
\subsubsection{Peru Specific Comments}\label{peru-specific-comments}}

MELQO data was merged with the Peru school frame in order to optimally
stratify. We stratified on the basis of urban/rual and department. There
are 25 departments in Peru. In 2017, Peru conducted an examination of
around 4,500 children between 5 and 8 years old, with a median age of 6.
The MELQO exam is quite similar to our ECD examination module. We are
able to use data from this 2017 survey to choose the number of schools
in each province optimally by calculating means and standard deviations
by province and feeding this information into the optimal stratification
algorithm. See
\url{https://cran.r-project.org/web/packages/SamplingStrata/vignettes/SamplingStrata.html}.
Provinces with low standard deviations among students in terms of their
MELQO development scores are allocated fewer schools compared to an
allocation that is simply based on population, and provinces with high
standard deviations are allocated more schools.

205 schools were chosen for our survey after optimally stratifying.

To see more detailed notes on our sampling approach in Peru, please see:
\url{https://github.com/worldbank/GEPD/blob/develop/Data/Peru/2019/Data/sampling/Sampling_Notes_Peru.html}

and

\url{https://github.com/worldbank/GEPD/blob/develop/Data/Peru/2019/Data/sampling/SamplingStrata_Peru_shared.R}

\hypertarget{jordan-specific-comments}{%
\subsubsection{Jordan Specific
Comments}\label{jordan-specific-comments}}

For our school survey, we select only schools that are supervised by the
Minsitry or Education or are Private schools. No schools supervised by
the Ministry of Defense, Ministry of Endowments, Ministry of Higher
Education , or Ministry of Social Development are included. This left us
with a sampling frame containing 3,330 schools, with 1297 private
schools and 2003 schools managed by the Minsitry of Education. The
schools must also have at least 3 grade 1 students, 3 grade 4 students,
and 3 teachers.

We oversampled Southern schools to reach a total of 50 Southern schools
for regional comparisons. Additionally, we oversampled Evening schools,
for a total of 40 evening schools.

To see more detailed notes on our sampling approach, please see:
\url{https://github.com/worldbank/GEPD/blob/develop/Data/Jordan/2019/Sampling/Sampling_Notes_Jordan.html}
and

\url{https://github.com/worldbank/GEPD/blob/develop/Data/Jordan/2019/Sampling/SamplingStrata_v2.R}

\hypertarget{mozambique-specific-comments}{%
\subsubsection{Mozambique Specific
Comments}\label{mozambique-specific-comments}}

Our sampled schools come from the list of schools surveyed by the 2018
SDI survey. Because we were supplementing data that was collected as
part of the 2018 SDI, we chose from among the schools sampled for this
survey. For public officials, the sampling was similar as described
above.

To see more detailed notes on our sampling approach, please see:
\url{https://github.com/worldbank/GEPD/blob/develop/Data/Mozambique/2019/Code/Sampling_Notes_Mozambique.html}

and

\url{https://github.com/worldbank/GEPD/blob/develop/Data/Mozambique/2019/Code/SamplingStrata_Mozambique.R}

\hypertarget{rwanda-specific-comments}{%
\subsubsection{Rwanda Specific
Comments}\label{rwanda-specific-comments}}

In order to visit two schools per day, we clustered at the sector level
choosing two schools per cluster. With a sample of 200 schools, this
means that we had to allocate 100 PSUs. We combined this clustering with
stratification by district and by the urban rural status of the schools.
The number of PSUs allocated to each stratum is proportionate to the
number of schools in each stratum (i.e.~the district X urban/rural
status combination).

To see more detailed notes on our sampling approach, please see:
\url{https://github.com/worldbank/GEPD/blob/develop/Data/Rwanda/2019/Data/Sampling/Sampling_Notes.html}

and

\url{https://github.com/worldbank/GEPD/blob/develop/Data/Rwanda/2019/Data/Sampling/SamplingStrata.R}

\hypertarget{punjab-specific-comments}{%
\subsubsection{Punjab Specific
Comments}\label{punjab-specific-comments}}

The survey in Pubjab is a combined effort of the Early Learning
Partnership project and of the Global Education Policy Dashboard
project.

Overall, we draw a sample of 200 public schools, 200 private schools and
200 public-private partnership (PPP) schools. We stratified by
urban/rural.

At this stage it is important to note, that there are certain districts
which we may not be able to visit due to security concerns, these are:

\begin{itemize}
\tightlist
\item
  Mianwali
\item
  Dera Ghazi Khan (DG Khan)
\item
  Rajan Pur
\item
  Bhakkar
\end{itemize}

We have removed these districts from the sampling frame.

Out of the 200 public schools to be surveyed we would like approximately
100 of these schools to be schools that are meeting ECE quality
standards (in the data set this corresponds to public\_strata==1). Out
of the remaining public schools to be sampled, 50 schools will be
schools that have ECE but do not meet quality standards
(public\_strata==2) and 50 will be schools that have no ECE at all, and
have only have katchi classes (public\_strata==3).

Due to operational constraints, we did not draw a random sample of all
schools at province level. We selected six districts for the survey (out
of 32). The survey team drew a convenience sample of 6 districts that is
representative of North, Central and South Punjab, which includes both
richer and poorer districts. A convenience sample was appropriate due to
security and operational constraints of working in Punjab. The selected
districts were:

\begin{itemize}
\tightlist
\item
  Attock
\item
  Faisalabad
\item
  Lahore
\item
  Muzaffargarh
\item
  Rahimyar Khan
\item
  Sargodha
\end{itemize}

In order to deal with potential refusals and closed schools, a set of
replacement schools was also drawn. Within the final strata, schools
were sampled proportional to size (number of total enrolled children in
pre-primary).

To see more detailed notes on our sampling approach, please see:
\url{https://github.com/worldbank/GEPD/blob/develop/Data/Pakistan_Punjab/2019/Code/Sampling/Sampling_Notes_Punjab.html}

and

\url{https://github.com/worldbank/GEPD/blob/develop/Data/Pakistan_Punjab/2019/Code/Sampling/SamplingStrata.R}

\hypertarget{power-considerations}{%
\subsection{Power Considerations}\label{power-considerations}}

In order to assess the likely detectable effect sizes for some of our
key indicators, we incorporated data from previous surveys that used a
similar methodology as the Global Education Policy Dashboard. We
examined the likely detectable effect sizes for the following
indicators: absenteeism, teacher content knowledge, 4th grade student
knowledge, Early Childhood Development, and Teacher pedagogical skills.

For absenteeism, we used data previously collected as part of the
Service Delivery Indicators (SDI) survey
(\url{https://www.sdindicators.org/}). Data came from the countries of
Afghanistan (SABER SD), Tanzania, Kenya, Mozambique, Nigeria, Uganda,
and Senegal. In each country, data on teacher presence was collected
using a very similar questionnaire. One difference is that for the SDI
survey absence was collected during a second unannounced visit to the
school. In our case, we notify the school that a team of enumerators
will be coming by the school over a two week period, but do not give
them the exact day. Setting aside this difference, we were able to
calculate the intra-class correlation (ICC) in absence for teachers
within the same school for each country and using these ICCs exame
expected detectable effect sizes by \# of schools sampled. With a
benchmark of around 200 schools, we expected to detect differences
between two years of between 0.06-0.08 percentage points in absenteeism.

\includegraphics{C:/Users/wb469649/Documents/Github/GEPD/Technical_Note/absence_power_schools.png}

For teacher content knowledge, we used the same data sources and a
similar methodology. Assuming a range of intra-class correlations which
were found in previous SDI countries, we expected to detect a change in
Teacher Content Knowledge of better than 0.05 standard deviations of
teacher knowledge over two years.

\includegraphics{C:/Users/wb469649/Documents/Github/GEPD/Technical_Note/know_power_schools.png}

For our 1st grade assessment, we used direct assessment data from the
MELQO survey of Kindergarten students in four countries: Ethiopia, Peru,
Mongolia, and Tanzania. ICCs for these countries ranged from 0.05 in
Mongolia to 0.43 in Tanzania. Assuming a sample size of 200, we would
expect to detect a change in Early Learner Knowledge of around 0.14 -0.2
standard deviations.

\includegraphics{C:/Users/wb469649/Documents/Github/GEPD/Technical_Note/ecd_power_schools.png}
Finally, for teacher pedagogical skills, we used prior data from the
TEACH survey in Mozambique and the Punjab province of Pakistan. Based on
data from these countries, we expected to detect a change in Teacher
Pedagogy of around 0.16-0.18 units on the 1-5 scale of teacher
pedagogical performance using our sample of 200 schools.\\
\includegraphics{C:/Users/wb469649/Documents/Github/GEPD/Technical_Note/teach_power_schools.png}

\hypertarget{validation-of-power-calculations-using-peru-data}{%
\subsection{Validation of Power Calculations using Peru
Data}\label{validation-of-power-calculations-using-peru-data}}

We can begin to test the assumptions of our power analysis by comparing
the precision implied in our power calculations to data actually
collected from our Survey in Peru in 2019. Our Peru sample is discussed
in greater detail above, but briefly we visited 205 schools. While our
power analyis was primarily concerned with the detectable effect size
over a two year period, we can still use data from one year of data
collection to assess whether our projects were on target.

Our power analysis aimed to assess the detectable effect size across two
years. Making an assumption that the sampling variance for our
indicators is constant across time, we can project the detectable effect
size under this assmption. Let the following be the standard error of
the difference in means of one of our indicators, where \(\bar X_j\) is
the sample mean for year j and \(\hat \sigma^2_j\) is the sample
variance for year j.

\(SE(\bar X_1-\bar X_0)=\sqrt(\hat \sigma^2_1/N_1 + \hat \sigma^2_0/N0)\)

and if we assume equal variances \(\sigma^2_1=\sigma^2_0\) and that the
sample sizes will be the same \(N_1=N_2\), the our Standard Error of the
difference of the means of the two samples is:

\(SE(X_1-X_0)=\sqrt(2* \hat \sigma^2_0/N0)\)

For teacher absenteeism, we were able to assess the absence of 1461
teachers in our sampled schools. This produced a mean absenteeism rate
of 90.68 and a standard error of 0.98. Therefore the standard error for
our two mean difference is just \(\sqrt 2\) times this standard error.
Therefore, our standard error for our two mean difference will be 1.39

Finally, to determine our detectable effect, we will be able to reject
the null of an identical mean for teacher absenteeism if our absenteeism
measure in the second Peru survey exceeds plus or minus: 2.73 percentage
points from our original mean of 90.68 .

This actually suggests that in the case of Peru, our forecast of
detectable effect sizes were too pessimistic, as we were expecting
somewhere in the range of 6-8 percentage points. This could be
attributed to lower than typical intra-class correlations in Peru for
absence or to improved precision resulting from our optimal
stratification technique.

For teacher content knowledge, we will omit the same full exercise as
with absence, but will only say that we will be able to reject the null
of an identical mean for teacher content knowledge if our measure in the
second Peru survey exceeds plus or minus: 5.25 percentage points from
our original mean.

For our Early Childhood assessment, can say that we will be able to
reject the null of an identical mean for ECE if our measure in the
second Peru survey exceeds plus or minus: 8.02 percentage points from
our original mean.

\hypertarget{full-tabulation-of-means-and-confidence-intervals-for-peru}{%
\subsection{Full Tabulation of Means and Confidence Intervals for
Peru}\label{full-tabulation-of-means-and-confidence-intervals-for-peru}}

Means and 95\% Confidence Intervals are available for all of our primary
indicators collected using our surveys below.

\begin{longtable}[]{@{}llrlrlrl@{}}
\caption{Table 1: Summary Statistics and Confidence Intervals for
Dashboard Indicators - Peru 2019}\tabularnewline
\toprule
Indicator & Value Range & Overall Mean & 95\% CI & Urban Mean & 95\% CI
& Rural Mean & 95\% CI\tabularnewline
\midrule
\endfirsthead
\toprule
Indicator & Value Range & Overall Mean & 95\% CI & Urban Mean & 95\% CI
& Rural Mean & 95\% CI\tabularnewline
\midrule
\endhead
Proficiency on GEPD Assessment & 0-100 & 51.54 & {[}47.69, 55.39{]} &
57.17 & {[}52.15, 62.19{]} & 27.64 & {[}22.6, 32.68{]}\tabularnewline
Student Attendance & 0-100 & 93.01 & {[}91.58, 94.44{]} & 93.34 &
{[}91.52, 95.16{]} & 91.63 & {[}88.86, 94.4{]}\tabularnewline
Teacher Effort & 0-100 & 90.68 & {[}88.75, 92.61{]} & 90.69 & {[}88.27,
93.11{]} & 90.65 & {[}86.83, 94.47{]}\tabularnewline
Teacher Content Knowledge & 0-100 & 38.69 & {[}34.98, 42.4{]} & 40.90 &
{[}35.64, 46.16{]} & 29.19 & {[}24.1, 34.28{]}\tabularnewline
Capacity for Learning & 0-100 & 52.65 & {[}46.98, 58.32{]} & 58.44 &
{[}50.33, 66.55{]} & 27.76 & {[}21.64, 33.88{]}\tabularnewline
Basic Inputs & 0-5 & 4.06 & {[}3.93, 4.19{]} & 4.13 & {[}3.94, 4.32{]} &
3.78 & {[}3.59, 3.97{]}\tabularnewline
Basic Infrastructure & 0-5 & 3.75 & {[}3.62, 3.88{]} & 3.99 & {[}3.84,
4.14{]} & 2.74 & {[}2.55, 2.93{]}\tabularnewline
Operational Management & 1-5 & 4.30 & {[}4.24, 4.36{]} & 4.31 & {[}4.22,
4.4{]} & 4.26 & {[}4.18, 4.34{]}\tabularnewline
Instructional Leadership & 1-5 & 3.70 & {[}3.58, 3.82{]} & 3.77 &
{[}3.6, 3.94{]} & 3.42 & {[}3.24, 3.6{]}\tabularnewline
Principal Knowledge of School & 1-5 & 3.62 & {[}3.43, 3.81{]} & 3.57 &
{[}3.3, 3.84{]} & 3.81 & {[}3.55, 4.07{]}\tabularnewline
Principal Management Skills & 1-5 & 4.64 & {[}4.58, 4.7{]} & 4.68 &
{[}4.6, 4.76{]} & 4.50 & {[}4.37, 4.63{]}\tabularnewline
Policy Lever (Teaching) - Attraction & 1-5 & 3.57 & {[}3.52, 3.62{]} &
3.57 & {[}3.5, 3.64{]} & 3.59 & {[}3.52, 3.66{]}\tabularnewline
Policy Lever (Teaching) - Selection \& Deployment & 1-5 & 3.65 &
{[}3.57, 3.73{]} & 3.62 & {[}3.5, 3.74{]} & 3.78 & {[}3.65,
3.91{]}\tabularnewline
Policy Lever (Teaching) - Support & 1-5 & 3.06 & {[}2.96, 3.16{]} & 3.05
& {[}2.91, 3.19{]} & 3.11 & {[}2.97, 3.25{]}\tabularnewline
Policy Lever (Teaching) - Evaluation & 1-5 & 3.96 & {[}3.88, 4.04{]} &
4.00 & {[}3.88, 4.12{]} & 3.76 & {[}3.65, 3.87{]}\tabularnewline
Policy Lever (Teaching) - Monitoring \& Accountability & 1-5 & 2.80 &
{[}2.73, 2.87{]} & 2.81 & {[}2.71, 2.91{]} & 2.75 & {[}2.65,
2.85{]}\tabularnewline
Policy Lever (Teaching) - Intrinsic Motivation & 1-5 & 4.59 & {[}4.55,
4.63{]} & 4.60 & {[}4.55, 4.65{]} & 4.54 & {[}4.48,
4.6{]}\tabularnewline
Policy Lever (Inputs \& Infrastructure) - Standards & 1-5 & 3.75 &
{[}3.56, 3.94{]} & 3.66 & {[}3.38, 3.94{]} & 4.14 & {[}3.89,
4.39{]}\tabularnewline
Policy Lever (Inputs \& Infrastructure) - Monitoring & 1-5 & 3.68 &
{[}3.56, 3.8{]} & 3.71 & {[}3.55, 3.87{]} & 3.56 & {[}3.37,
3.75{]}\tabularnewline
Policy Lever (School Management) - Clarity of Functions & 1-5 & 4.95 &
{[}4.91, 4.99{]} & 4.96 & {[}4.91, 5.01{]} & 4.91 & {[}4.84,
4.98{]}\tabularnewline
Policy Lever (School Management) - Attraction & 1-5 & 4.39 & {[}4.31,
4.47{]} & 4.45 & {[}4.35, 4.55{]} & 4.12 & {[}4, 4.24{]}\tabularnewline
Policy Lever (School Management) - Selection \& Deployment & 1-5 & 4.38
& {[}4.27, 4.49{]} & 4.38 & {[}4.22, 4.54{]} & 4.42 & {[}4.27,
4.57{]}\tabularnewline
Policy Lever (School Management) - Support & 1-5 & 3.52 & {[}3.38,
3.66{]} & 3.61 & {[}3.42, 3.8{]} & 3.14 & {[}2.92,
3.36{]}\tabularnewline
Policy Lever (School Management) - Evaluation & 1-5 & 4.36 & {[}4.26,
4.46{]} & 4.42 & {[}4.28, 4.56{]} & 4.12 & {[}3.95,
4.29{]}\tabularnewline
Politics \& Bureaucratic Capacity - National Learning Goals & 1-5 & 3.01
& {[}2.92, 3.1{]} & NA & NA & NA & NA\tabularnewline
Politics \& Bureaucratic Capacity - Mandates \& Accountability & 1-5 &
4.17 & {[}4.09, 4.25{]} & NA & NA & NA & NA\tabularnewline
Politics \& Bureaucratic Capacity - Quality of Bureaucracy & 1-5 & 3.96
& {[}3.89, 4.03{]} & NA & NA & NA & NA\tabularnewline
Politics \& Bureaucratic Capacity - Impartial Decision-Making & 1-5 &
3.33 & {[}3.21, 3.45{]} & NA & NA & NA & NA\tabularnewline
\bottomrule
\end{longtable}

\hypertarget{survey-weights}{%
\subsection{Survey Weights}\label{survey-weights}}

Survey weights are constructed using our original sampling frame for a
country. School Weights are the inverse probability that a 4th grade
student in the school is randomly selected for the school survey. This
is calculated by summing the total number of students in each particular
school and dividing by the total number of 4th grade students in the
sampling frame. In cases where we have large amounts of non-responses or
refusals to enter the school, we will add an adjustment term to account
for non-response. This could happen for instance, if an originally
sampled school was selected but refused and the two replacement schools
for that school also refused.

For student level or teacher level data, we will also create an
adjustment for the random selection of classrooms, teachers, or students
that takes place within schools. In order to form these weights at the
individual level, the school weight is multiplied by the number of units
sampled (five teachers for the teacher interview, 10 for the teacher
absence module, one for the class of 4th graders, and three for the
three randomly selected 1st grade students.).

\hypertarget{data-collection}{%
\section{Data Collection}\label{data-collection}}

Below, we will briefly discuss the data collection protocols and
processes used by our team. In short, we use Survey Solutions, which is
a tablet based, free, open source survey tool designed by the World
Bank. The data flow then is as follows:

\begin{enumerate}
\def\labelenumi{\arabic{enumi}.}
\tightlist
\item
  Data is collected by an enumerator\\
\item
  The enumerator or their supervisor then uploads the collected data to
  our secure server (typically a server generated by the Survey
  Solutions team)\\
\item
  The data is encrypted on the server\\
\item
  The Global Education Policy Dashboard Team then downloads the
  encrypted data using the Survey Solutions API
\item
  Our data is stored in a secure folder
\item
  We clean the data, create unique IDs for students, teachers,
  principals, schools, and public officials using cryptographic hashing
\item
  A number of data quality checks are run to minimize missing values,
  identify enumerator errors, and ensure accurate coding\\
\item
  The data is then anonymized removing all PII information\\
\item
  Our final indicators are then aggregated to the National level, as
  well as offering breakdowns by Urban/Rural and Gender\\
\item
  This aggregated data is then uploaded to the World Bank open data
  platform Edstats\\
\item
  The Global Education Policy Dashboard website then pulls the data from
  EdStats using the EdStats API system
\end{enumerate}

\hypertarget{survey-solutions}{%
\subsection{Survey Solutions}\label{survey-solutions}}

Survey Solutions is a free, open source data collection platform
designed by World Bank staff.

The raw data will be hosted on a secure server, which is certified by
several dozen well known international bodies, including the EU Data
Protection Directive. More information on the certifications can be
found here:
\url{https://support.mysurvey.solutions/faq/how-secure-is-the-world-bank-cloud-/}.

\hypertarget{anonymization}{%
\subsection{Anonymization}\label{anonymization}}

Immediately following download of data from Survey Solutions, we remove
the following and save as an anonymized version of the data: 1.Drop
Enumerator name\\
2. Drop school name, address, official school codes (EMIS codes)\\
3. Drop principal name, phone numbers\\
4. Drop Geo-code info\\
5. Drop unique responses, such as when respondent is asked to specify
other as a choice

The following are produced: 1. Produce Crypto-hashed School ID, Province
ID, District ID

We convert the following variables into categorical data: 2. Respondent
age 3. Year began teaching 4. Number of students in school/class 5.
Top-coded variables: 6. Educational degrees

\hypertarget{data-quality-checks}{%
\section{Data Quality Checks}\label{data-quality-checks}}

While survey data is collected, we run daily quality checks to ensure
the data is high quality. These include both built in checks available
through the survey solutions HeadQuarters app, and also a custom
designed data quality check tool.

Survey Solutions has several built in quality checks. These include a
mapping tool based on the geocoordinates enterred by the enumerators
during data collection and an interview management system where
supervisors or the headquarters can see whether all questions that
should have been completed by the enumerator are completed. Supervisors
then have the option to send feedback to the enumerator and reject the
interview until the data is sucessfully collected.

Additionally, our team has designed a set of custom tools for our Survey
of Public Officials and School Survey. All code for our data quality
checks can be found on our Global Education POlicy Dashboard github site
(\url{https://github.com/worldbank/GEPD}). These tools allow the field
team to calculate our final primary indicators in real time and check
for any anomalies including missing values, data enterred at strange
times (e.g.~after regular business hours), and to look at data collected
for specific enumerators.

\hypertarget{indicator-construction}{%
\section{Indicator Construction}\label{indicator-construction}}

\begin{longtable}[]{@{}lll@{}}
\caption{Table 2: High Level Description of How Each Indicator is
Scored}\tabularnewline
\toprule
Indicator Name & Values & How Indicator is Scored\tabularnewline
\midrule
\endfirsthead
\toprule
Indicator Name & Values & How Indicator is Scored\tabularnewline
\midrule
\endhead
Proficiency on GEPD Assessment & 0-100 & The fraction of students
scoring at least 20/24 points on 4th grade language and 14/17 points on
the math student assessment. Our team consulted several content experts
to advise on how many of our math and language items a minimally
proficient 4th grade student should be able to get correct to decide
these thresholds.\tabularnewline
Participation & 0-100 & Based on adjusted net enrollment data. Not based
on survey data\tabularnewline
Teacher Effort & 0-100 & School survey. Percent of teachers present.
Teacher is coded absent if they are: - not in school - in school but
absent from the class.\tabularnewline
Teacher Content Knowledge & 0-100 & School survey. The fraction of
teachers scoring at least 80\% on the teacher content knowledge exam on
mathematics and language.\tabularnewline
Teacher Pedagogical Skills & 1-5 & School survey. Based on TEACH ratings
from classroom observations\tabularnewline
Basic Inputs & 0-5 & School survey. Total score is the sum of whether a
school has: - Functional blackboard - Pens, pencils, exercise books -
Textbooks - Fraction of students in class with a desk - Used ICT in
class and have access to ICT in the school.\tabularnewline
Basic Infrastructure & 0-5 & School survey. Total score is the sum of
whether a school has: - Access to adequate drinking water -Functional
toilets that are separate for boys/girls, private, useable, and have
hand washing facilities - Electricity - Internet - School is accessible
for those with disabilities (road access, a school ramp for wheelchairs,
an entrance wide enough for wheelchairs, ramps to classrooms where
needed, accessible toilets, and disability screening for seeing,
hearing, and learning disabilities with partial credit for having 1 or 2
or the 3).)\tabularnewline
Capacity for Learning & 0-100 & School survey. The average of Fraction
correct on the literacy, numeracy, socio-emotional, and executive
functioning components of the early childhood assessment given to
students in school.\tabularnewline
Student Attendance & 0-100 & School Survey. Percent of 4th grade
students who are present during an unannounced visit.\tabularnewline
Operational Management & 1-5 & School Survey. Principals/head teachers
are given two vignettes: - One on solving the problem of a hypothetical
leaky roof - One on solving a problem of inadequate numbers of
textbooks. Each vignette is worth 2 points. The indicator will measure
two things: presence of functions and quality of functions. In each
vignette: - 0.5 points are awarded for someone specific having the
responsibility to fix - 0.5 point is awarded if the school can fully
fund the repair, 0.25 points is awarded if the school must get partial
help from the community, and 0 points are awarded if the full cost must
be born by the community - 1 point is awarded if the problem is fully
resolved in a timely manner, with partial credit given if problem can
only be partly resolved.\tabularnewline
Instructional Leadership & 1-5 & School survey. Total score starts at 1
and points added are the sum of whether a teacher has: - Had a classroom
observation in past year - Had a discussion based on that observation
that lasted longer than 30 min - Received actionable feedback from that
observation - Teacher had a lesson plan and discussed it with another
person\tabularnewline
Principal Knowledge of School & 1-5 & The aim of this indicator is to
measure the extent to which principals have the knowledge about their
own schools that is necessary for them to be effective managers. A score
from 1 to 5 capturing the extent to which the principal is familiar with
certain key aspects of the day-to-day workings of the school (in schools
that have principals). Principal receives points in the following way: -
5 points. Principal gets all 90-100\% of questions within accuracy
bounds (defined below). - 4 points. Principal gets 80-90\% of question
within accuracy bounds. - 3 points. Principal gets 70-80\% of question
within accuracy bounds. - 2 points. Principal gets 60-70\% of question
within accuracy bounds. - 1 points. Principal gets under 60\% of
question within accuracy bounds. Accuracy bounds for each question. Got
80\% of teachers correct or within 1 teacher/student for each of the
following: - Out of these XX teachers, how many do you think would be
able to correctly add triple digit numbers (i.e.~343+215+127)? - Out of
these XX teachers, how many do you think would be able to correctly to
multiply double digit numbers (i.e.~37 x 13)? - Out of these XX
teachers, how many do you think would be able to complete sentences with
the correct world (i.e.~The accident \_\_\_\_\_ (see, saw, had seen, was
seen) by three people)? - Any of these XX teachers have less than 3
years of experience? - Out of these XX teachers, which ones have less
than 3 years of experience as a teacher? Got 80\% of students correct or
within 2 students for each of the following: - In the selected 4th grade
classroom, how many of the pupils have the relevant textbooks? Must
identify whether or not blackboard was working in a selected 4th grade
classroom.\tabularnewline
Principal Management Skills & 1-5 & Score of 1-5 based on sum of
following: - 1 Point. School Goals Exists - 1 Point. School goals are
clear to school director, teachers, students, parents, and other members
of community (partial credit available) - 1 Point. Specific goals
related to improving student achievement ( improving test scores,
improving pass rates, reducing drop out, reducing absenteeism, improving
pedagogy, more resources for infrastructure, more resources for inputs)
- 1 Point. School has defined system to measure goals (partial credit
available)\tabularnewline
Policy Lever (Teaching) - Attraction & 1-5 & In the school survey, a
number of De Facto questions on teacher attraction are asked. 0.8 points
is awarded for each of the following: - 0.8 Points. Teacher satisfied
with job - 0.8 Points. Teacher satisfied with status in community - 0.8
Points. Would better teachers be promoted faster? - 0.8 Points. Do
teachers receive bonuses? - 0.8 Points. One minus the fraction of months
in past year with a salary delay.\tabularnewline
Policy Lever (Teaching) - Selection \& Deployment & 1-5 & School Survey.
The De Facto portion of the Teacher Selection and Deployment Indicator
considers two issues: how teachers are selected into the profession and
how teachers are assigned to positions (transferred) once in the
profession. Research shows that degrees and years of experience explain
in little variation in teacher quality, so more points are assigned for
systems that also base hiring on content knowledge or pedagogical skill.
2 points are available for the way teachers are selected and 2 points
are available for deployment. Selection - 0 Points. None of the below -
1 point. Teachers selected based on completion of coursework,
educational qualifications, graduating from tertiary program (including
specialized programs), selected based on experience - 2 points. Teacher
recruited based on passing written content knowledge test, passed
interview stage assessment, passed an assessment conducted by supervisor
based on practical experience, conduct during mockup class. Deployment -
0 Points. None of the below - 1 point. Teachers deployed based on years
of experience or job title hierarchy or other criteria - 2 points.
Teacher deployed based on performance assessed by school authority,
colleagues, or external evaluator, results of interview.\tabularnewline
Policy Lever (Teaching) - Support & 1-5 & School survey. Our teaching
support indicator asks teachers about participation and the experience
with several types of formal/informal training: Pre-Service (Induction)
Training: - 0.5 Points. Had a pre-service training - 0.5 Points. Teacher
reported receiving usable skills from training Teacher practicum (teach
a class with supervision) - 0.5 Points. Teacher participated in a
practicum - 0.5 Points. Practicum lasted more than 3 months and teacher
spent more than one hour per day teaching to students. In-Service
Training: - 0.5 Points. Had an in-service training - 0.25 Points.
In-service training lasted more than 2 total days - 0.125 Points. More
than 25\% of the in-service training was done in the classroom. - 0.125
Points. More than 50\% of the in-service training was done in the
classroom. Opportunities for teachers to come together to share ways of
improving teaching: - 1 Point if such opportunities
exist.\tabularnewline
Policy Lever (Teaching) - Evaluation & 1-5 & School survey. This policy
lever measures whether there is a teacher evaluation system in place,
and if so, the types of decisions that are made based on the evaluation
results. Score is the sum of the following: - 1 Point. Was teacher
formally evaluated in past school year? - 1 Point total. 0.2 points for
each of the following: Evaluation included evaluation of attendance,
knowledge of subject matter, pedagogical skills in the classroom,
students' academic achievement, students' socio-emotional development -
1 Point. Consequences exist if teacher receives 2 or more negative
evaluations - 1 Point. Rewards exist if teacher receives 2 or more
positive evaluations\tabularnewline
Policy Lever (Teaching) - Monitoring \& Accountability & 1-5 & School
Survey. This policy lever measures the extent to which teacher presence
is being monitored, whether attendance is rewarded, and whether there
are consequences for chronic absence. Score is the sum of the following:
- 1 Point. Teachers evaluated by some authority on basis of absence. - 1
Point. Good attendance is rewarded. - 1 Point. There are consequences
for chronic absence (more than 30\% absence). - 1 Point. One minus the
fraction of teachers that had to miss class because of any of the
following: collect paycheck, school administrative procedure, errands or
request of the school district office, other administrative
tasks.\tabularnewline
Policy Lever (Teaching) - Intrinsic Motivation & 1-5 & School Survey.
This lever measures whether teachers are intrinsically motivated to
teach. The question(s) aim to address this phenomenon by measuring the
level of intrinsic motivation among teachers as well as teacher values
that may be relevant for ensuring that the teacher is motivated to focus
on all children and not just some. Score is sum of the following - Max
0.8 Points.. Average response to battery of questions on whether
teachers considers it acceptable to be absent in certain situations.
Average response scores scaled to be continuous score between 0-1. - Max
10.8 Points.. Average response to battery of questions on whether
teachers consider some students to be more deserving of attention.
Average response scores scaled to be continuous score between 0-1. - Max
0.8 point. Average response to battery of questions on teachers growth
mindset regarding students. Average response scores scaled to be
continuous score between 0-1. - 0.8 Points. Binary response to whether
teacher said they became teacher, because teaching offered steady
career. Teacher scored 0 points if they reported wanting to teach for a
steady career. -0.8 Points. Binary response of whether there is a
probationary period for teachers.\tabularnewline
Policy Lever (Inputs \& Infrastructure) - Standards & 1-5 & School
Survey. This lever measures the extent to which there is a monitoring
system in place to ensure that the inputs that must be available at the
schools are in fact available at the schools. This set of questions will
include three aspects: - 0.5 Points for each. Are there standards in
place to monitor blackboard and chalk, pens and pencils, basic classroom
furniture, computers, textbooks, exercise books, toilets, electricity,
drinking water, accessibility for those with disabilities? (partial
credit available)\tabularnewline
Policy Lever (Inputs \& Infrastructure) - Monitoring & 1-5 & School
Survey. This lever measures the extent to which there is a monitoring
system in place to ensure that the inputs that must be available at the
schools are in fact available at the schools. This set of questions will
include three aspects: - 1.5 Point. Are all input items (functioning
blackboard, chalk, pens, pencils, textbooks, exercise books in 4th grade
classrooms, basic classroom furniture, and at least one computer in the
schools) being monitored? (partial credit available) - 1.5 Point. Are
all infrastructure items (functioning toilets, electricity, drinking
water, and accessibility for people with disabilities) being monitored?
(partial credit available) - 1 Point. Is the community involved in the
monitoring?\tabularnewline
Policy Lever (Learners) - Nutrition Programs & 1-5 & Policy
Survey.\tabularnewline
Policy Lever (Learners) - Health & 1-5 & Policy Survey.\tabularnewline
Policy Lever (Learners) - Center-Based Care & 1-5 & Policy
Survey.\tabularnewline
Policy Lever (Learners) - Caregiver Capacity - Financial Capacity & 1-5
& Policy Survey.\tabularnewline
Policy Lever (Learners) - Caregiver Capacity - Skills Capacity & 1-5 &
Policy Survey.\tabularnewline
Policy Lever (School Management) - Clarity of Functions & 1-5 & Policy
Survey.\tabularnewline
Policy Lever (School Management) - Attraction & 1-5 & This policy lever
measures whether the right candidates are being attracted to the
profession of school principals. The questions will aim to capture the
provision of benefits to attract and maintain the best people to serve
as principals. Scoring: - For now, score is between 1-5 based on how
satisfied the principal is with status in community. We will also add in
component based on Principal salaries.\tabularnewline
Policy Lever (School Management) - Selection \& Deployment & 1-5 & This
policy lever measures whether the right candidates being selected. These
questions will probe what the recruitment process is like to ensure that
these individuals are getting the positions. The question would
ultimately be based on: 1) there is a standard approach for selecting
principals, 2) that approach relies on professional/academic
requirements, and 3) those requirements are common in practice. Scoring:
- 1 (lowest score) Most important factor is political affiliations or
ethnic group. - 2 Political affiliations or ethnic group is a
consideration, but other factors considered as well. - 3 Most important
factor is years of experience, good relationship with owner/education
department, and does not factor in quality teaching, demonstrated
management qualities, or knowledge of local community. - 4 Quality
teaching, demonstrated management qualities, or knowledge of local
community is a consideration in hiring, but not the most important
factor - 5 Quality teaching, demonstrated management qualities, or
knowledge of local community is the most important factor in
hiring.\tabularnewline
Policy Lever (School Management) - Support & 1-5 & This policy lever
measures the extent to which principals receive training and/or exposure
to other professional opportunities that could help them be better
school leaders. The questions aim to figure out if such programs are
provided, and if they are, their level of quality. Scoring (sum of
components below): - 1 Point. Principal has received formal training on
managing school. - 1/3 Point. Had management training for new
principals. - 1/3 Point. Had management in-service training. - 1/3
Point. Had mentoring/coaching by experienced principals. - 1 Point. Have
used skills gained at training. - 1 Point. Principals offered training
at least once per year\tabularnewline
Policy Lever (School Management) - Evaluation & 1-5 & School Survey.
This policy lever measures the extent to which principal performance is
being monitored and enforced via accountability measures. The idea is
that the indicator will be based on: 1) there is a legislation outlining
the need to monitor, 2) principals are being evaluated, 3) principals
are being evaluated on multiple things, and 4) there the accountability
mechanisms in place. Score is the sum of the following: - 1 Point. Was
principal formally evaluated in past school year? - 1 Point. Evaluation
based on multiple components. Partial Credit available. 1/3 point for
just one component, 2/3 points for 2-5 components, 1 point for more than
5 components. - 1 Point. Consequences exist if principal receives 2 or
more negative evaluations - 1 Point. Rewards exist if principal receives
2 or more positive evaluations\tabularnewline
Politics \& Bureaucratic Capacity - Quality of Bureaucracy & 1-5 &
Survey of Public officials. This indicator assesses the quality of the
bureaucracy, which is the implementing machinery of the government for
achieving national learning goals and ensuring that the policies created
to promote learning are enforced. Average score (1 (worst) - 5 (best) )
on items given to public officials. Some specific comments: - On QB1q2 -
What is the average class size in a typical 4th-grade class of the
country? - I assigned 5 points if the guess was within 10\% of the
correct answer, 4 points if between 10-20\% of the correct answer, 3 if
between 20 and 30\%, 2 if between 30 and 40\%, and 1 if more than 40\%
away from correct answer. - On QB1q1 - What percent of their time do you
think teachers are absent without providing justification? - this
question is scored similar as above. - for QB4q2 - Imagine that when you
started your motivation was 100. What number would you say your
motivation was now relative to that? - We score over 120\% as a 5,
110-120\% as a 4, 100-110\% as a 3, 90-100 as a 2, and below 90 as a
1\tabularnewline
Politics \& Bureaucratic Capacity - Impartial Decision-Making & 1-5 &
Survey of Public officials. This module assesses the extent to which
bureaucrats implement policies in an impartial way, meaning that
decisions are free from political clientelism or undue influence from
any single interest group. Average score (1 (worst) - 5 (best)) on items
given to public officials.\tabularnewline
Politics \& Bureaucratic Capacity - Mandates \& Accountability & 1-5 &
Survey of Public officials. The aim of this indicator is to measure the
extent to which the mandates are clearly defined and allocated in the
legislation, as well as whether such allocation is reflected in
practice. Average score (1 (worst) - 5 (best)) on items given to public
officials.\tabularnewline
Politics \& Bureaucratic Capacity - National Learning Goals & 1-5 &
Survey of Public officials. The aim of this indicator is to capture the
extent to which there is a goal and/or strategy that encapsulates a
desire and a path to reach higher learning outcomes. This is what
ultimately drives the work of the ministry and district offices that
together, with the support of other stakeholders and civil society, work
towards achieving that goal. Average score (1 (worst) - 5 (best)) on
items given to public officials.\tabularnewline
Politics \& Bureaucratic Capacity - Financing & 1-5 & Policy
Survey.\tabularnewline
\bottomrule
\end{longtable}

\hypertarget{appendix}{%
\section{Appendix}\label{appendix}}

\hypertarget{detailed-indicator-construction}{%
\subsection{Detailed Indicator
Construction}\label{detailed-indicator-construction}}

Below is a list of the primary indicators and sub-indicators used to
construct our final indicators.

\begin{longtable}[]{@{}lll@{}}
\caption{Table 3: Detailed Documentation of Indicators \&
Sub-Indicators}\tabularnewline
\toprule
Variable & How Was Data Collected? & Details on Scoring\tabularnewline
\midrule
\endfirsthead
\toprule
Variable & How Was Data Collected? & Details on Scoring\tabularnewline
\midrule
\endhead
4th Grade Student Knowledge & 4th Grade Student Assessment & Fraction
correct on our 4th grade assessment\tabularnewline
4th Grade Math Knowledge & 4th Grade Student Assessment & Fraction
correct on our 4th grade math assessment\tabularnewline
4th Grade Literacy Knowledge & 4th Grade Student Assessment & Fraction
correct on our 4th grade language assessment\tabularnewline
4th Grade Student Proficiency & 4th Grade Student Assessment & Whether
or not student reached 34/41 items correct\tabularnewline
4th Grade Student Proficiency Literacy & 4th Grade Student Assessment &
Whether or not student reached 20/24 items correct\tabularnewline
4th Grade Student Proficiency Math & 4th Grade Student Assessment &
Whether or not student reached 14/17 items correct\tabularnewline
4th Grade Student Proficiency at 70\% threshold & 4th Grade Student
Assessment & 4th Grade Student Proficiency at 70\%
threshold\tabularnewline
4th Grade Student Proficiency at 75\% threshold & 4th Grade Student
Assessment & 4th Grade Student Proficiency at 75\%
threshold\tabularnewline
Student Attendance Rate & Student Roster and Classroom Observation & \#
of students present divided by \# according to class list\tabularnewline
Teacher Classroom Presence Rate & Teacher Roster and School Inspection &
Fraction of teachers out of 10 that were in class and school during
attendance check\tabularnewline
Teacher Classroom Absence Rate & Teacher Roster and School Inspection &
Fraction of teacher not in class or school during attendance
check\tabularnewline
Teacher School Absence Rate & Teacher Roster and School Inspection &
Fraction of teacher not in school during attendance check\tabularnewline
Teacher Content Proficiency & Teacher Assessment & Fraction of teachers
scoring 80\% of higher on overall content assessment\tabularnewline
Teacher Content Proficiency Literacy & Teacher Assessment & Fraction of
teachers scoring 80\% of higher on literacy content
assessment\tabularnewline
Teacher Content Proficiency Math & Teacher Assessment & Fraction of
teachers scoring 80\% of higher on math content
assessment\tabularnewline
Teacher Content Proficiency at 70\% threshold & Teacher Assessment &
Teacher Content Proficiency at 70\% threshold\tabularnewline
Teacher Content Proficiency at 75\% threshold & Teacher Assessment &
Teacher Content Proficiency at 75\% threshold\tabularnewline
Teacher Content Knowledge & Teacher Assessment & Average score at school
level on math and literacy portion of content exam\tabularnewline
Teacher Math Content Knowledge & Teacher Assessment & Fraction correct
on overall math assessment\tabularnewline
Teacher Literacy Content Knowledge & Teacher Assessment & Fraction
correct on overall literacy assessment\tabularnewline
Grammar & Teacher Assessment & Fraction correct on grammar task of
content assessment\tabularnewline
Cloze Task & Teacher Assessment & Fraction correct on cloze task of
content assessment\tabularnewline
Read Passage & Teacher Assessment & Fraction correct on read a passage
task of content assessment\tabularnewline
Arithmetic \& Number Relations & Teacher Assessment & Fraction correct
on arithmetic and number relations task of content
assessment\tabularnewline
Geometry & Teacher Assessment & Fraction correct on geometry task of
content assessment\tabularnewline
Interpret Data & Teacher Assessment & Fraction correct on interpretting
data task of content assessment\tabularnewline
1st Grade Assessment Score & 1st Grade Direct Assessment & Average of
the fraction correct in numeracy, literacy, socio-emotional, and
executive functioning\tabularnewline
1st Grade Numeracy Score & 1st Grade Direct Assessment & 1st Grade
Numeracy Fraction Correct\tabularnewline
1st Grade Literacy Score & 1st Grade Direct Assessment & 1st Grade
Literacy Fraction Correct\tabularnewline
1st Grade Executive Functioning Score & 1st Grade Direct Assessment &
1st Grade Executive Functioning Fraction Correct\tabularnewline
1st Grade Socio-Emotional Score & 1st Grade Direct Assessment & 1st
Grade Socio-Emotional Fraction Correct\tabularnewline
1st Grade Assessment Proficiency & 1st Grade Direct Assessment & Marked
as proficient if student could score at least 80\% of items
correct\tabularnewline
1st Grade Numeracy Proficiency & 1st Grade Direct Assessment & Marked as
proficient if student could score at least 90\% of math items
correct\tabularnewline
1st Grade Literacy Proficiency & 1st Grade Direct Assessment & Marked as
proficient if student could score at least 80\% of literacyitems
correct\tabularnewline
1st Grade Executive Functioning Proficiency & 1st Grade Direct
Assessment & Marked as proficient if student could score at least 70\%
of executive functiong items correct\tabularnewline
1st Grade Socio-Emotional Proficiency & 1st Grade Direct Assessment &
Marked as proficient if student could score at least 80\% of
socio-emotional items correct\tabularnewline
Inputs & Classroom Observation & Total of whether there is functional
blackboard, materials, desks, and ICT\tabularnewline
Functioning Blackboard (Based on classroom observation: Is there a
blackboard, Is there chalk, and Is there sufficient light & Classroom
Observation & Functioning Blackboard (Based on classroom observation: Is
there a blackboard, Is there chalk, and Is there sufficient
light\tabularnewline
Classroom Materials (Pens, Pencils, Exercise Books) & Classroom
Observation & \# of Classroom Materials (Pens, Pencils, Exercise Books)
divided by \# of students in class\tabularnewline
Textbooks & Classroom Observation & \# of Classroom Textbooks divided by
\# of students in class\tabularnewline
Desks & Classroom Observation & \# of Classroom Desks divided by \# of
students in class\tabularnewline
ICT Usage & Teacher Inverview & Whether or not teacher reported using
ICT\tabularnewline
ICT Access & School Inspection & Whether tablets/computers were working
and had internet connectivity\tabularnewline
Infrastructure & School Inspection & Infrastructure\tabularnewline
Clean Drinking Water & School Inspection & Enumerators assess whether
water comes from piped water, protected well, packaged bottle water, or
tanker truck. No credit for unprotected sources\tabularnewline
Functioning Toilets & School Inspection & Toilets exist, separate for
boys/girls, clean, private, useable, handwashing
available\tabularnewline
Internet & School Inspection & 1 point if internet working, 0.5 if
doesn't work well, 0 if not at all\tabularnewline
Electricity & School Inspection & Enuermator assessment of whether
electricity worked in randomly selected class\tabularnewline
Disability Accessibility & School Inspection & Accessibility from road,
has ramp with accessible entrance for school and
classrooms\tabularnewline
Operational Management & Principal Interview & Operational
Management\tabularnewline
Operational Management - Vignette 1 & Principal Interview & Leaky roof
vignette: 0.5 points are awarded for someone specific having the
responsibility to fix,0.5 point is awarded if the school can fully fund
the repair, 0.25 points is awarded if the school must get partial help
from the community, and 0 points are awarded if the full cost must be
born by the community, 1 point is awarded if the problem is fully
resolved in a timely manner, with partial credit given if problem can
only be partly resolved.\tabularnewline
Operational Management - Vignette 2 & Principal Interview & Inadequate
number of textbooks vignette: 0.5 points are awarded for someone
specific having the responsibility to fix,0.5 point is awarded if the
school can fully address, 0.25 points is awarded if the school must get
partial help from the community, and 0 points are awarded if the full
cost must be born by the community, 1 point is awarded if the problem is
fully resolved in a timely manner, with partial credit given if problem
can only be partly resolved.\tabularnewline
Teacher Intrinsic Motivation & Teacher Interview & Teacher Intrinsic
Motivation\tabularnewline
Scores on set of questions to teachers on whether it is ever acceptable
to be absent & Teacher Interview & Scores on set of questions to
teachers on whether it is ever acceptable to be absent\tabularnewline
Scores on set of questions to teachers on whether some students deserve
more attention than others & Teacher Interview & Scores on set of
questions to teachers on whether some students deserve more attention
than others\tabularnewline
Scores on questions related to growth mindset & Teacher Interview &
Scores on questions related to growth mindset\tabularnewline
Motivation to teach based on whether teacher joined teaching just for
job security & Teacher Interview & Motivation to teach based on whether
teacher joined teaching just for job security\tabularnewline
Instructional Leadership & Teacher Interview & Instructional
Leadership\tabularnewline
Whether or not classroom has been observed & Teacher Interview & Whether
or not classroom has been observed\tabularnewline
Whether observation took place in last 12 months & Teacher Interview &
Whether observation took place in last 12 months\tabularnewline
Whether teacher discussed classroom observation with principal/head
teacher & Teacher Interview & Whether teacher discussed classroom
observation with principal/head teacher\tabularnewline
Whehter feedback was given on classroom observation and converation
lasted more than 30 min & Teacher Interview & Whehter feedback was given
on classroom observation and converation lasted more than 30
min\tabularnewline
Whether teacher had feedback on lesson plan & Teacher Interview &
Whether teacher had feedback on lesson plan\tabularnewline
Principal Knowledge of School & Principal Interview & Principal
Knowledge of School\tabularnewline
Correct on \# of Teachers correct on Triple Digit Addition & Principal
Interview & Correct on \# of Teachers correct on Triple Digit
Addition\tabularnewline
Correct on \# of Teachers correct on Double Digit Multiplication &
Principal Interview & Correct on \# of Teachers correct on Double Digit
Multiplication\tabularnewline
Correct on \# of Teachers correct on Completing Sentence Question &
Principal Interview & Correct on \# of Teachers correct on Completing
Sentence Question\tabularnewline
Correct on \# of Teachers Under 3 Years Experience & Principal Interview
& Correct on \# of Teachers Under 3 Years Experience\tabularnewline
Correct on \# of Students with Textbooks & Principal Interview & Correct
on \# of Students with Textbooks\tabularnewline
Correct on Functional Blackboard & Principal Interview & Correct on
Functional Blackboard\tabularnewline
Principal Management Skills & Principal Interview & Principal Management
Skills\tabularnewline
According to Principal: School Goals Exist & Principal Interview &
According to Principal: School Goals Exist\tabularnewline
According to Principal: to Principal: School Goals Clear & Principal
Interview & According to Principal: to Principal: School Goals
Clear\tabularnewline
According to Principal: School Goals Relevant & Principal Interview &
According to Principal: School Goals Relevant\tabularnewline
According to Principal: School Goals Measured & Principal Interview &
According to Principal: School Goals Measured\tabularnewline
Teacher Attraction (De Facto) & Teacher Interview & Teacher Attraction
(De Facto)\tabularnewline
Teacher Satisfied with Job & Teacher Interview & Teacher Satisfied with
Job\tabularnewline
Teachers Satisfied with Status in Community & Teacher Interview &
Teachers Satisfied with Status in Community\tabularnewline
Teacher perceives that better teachers more likely to be promoted &
Teacher Interview & Teacher perceives that better teachers more likely
to be promoted\tabularnewline
Teacher received bonus in past year & Teacher Interview & Teacher
received bonus in past year\tabularnewline
Salary Delays reported by teacher & Teacher Interview & Salary Delays
reported by teacher\tabularnewline
Teacher Selection \& Deployment (De Facto) & Teacher Interview & Teacher
Selection \& Deployment (De Facto)\tabularnewline
Teacher recruited based on passing written content knowledge test,
passed interview stage assessment, passed an assessment conducted by
supervisor based on practical experience, conduct during mockup class &
Teacher Interview & Teacher recruited based on passing written content
knowledge test, passed interview stage assessment, passed an assessment
conducted by supervisor based on practical experience, conduct during
mockup class\tabularnewline
Teacher deployed based on performance assessed by school authority,
colleagues, or external evaluator, results of interview. & Teacher
Interview & Teacher deployed based on performance assessed by school
authority, colleagues, or external evaluator, results of
interview.\tabularnewline
Teacher Support (De Facto) & Teacher Interview & Teacher Support (De
Facto)\tabularnewline
Had pre-service training and teacher reported finding it useful &
Teacher Interview & Had pre-service training and teacher reported
finding it useful\tabularnewline
Had a practicum and training lasted multiple days & Teacher Interview &
Had a practicum and training lasted multiple days\tabularnewline
Had in-service training that lasted more than 2 days and had a component
that took place in classroom & Teacher Interview & Had in-service
training that lasted more than 2 days and had a component that took
place in classroom\tabularnewline
Whether teacher reported having opportunities to come together and share
ways of improving teaching & Teacher Interview & Whether teacher
reported having opportunities to come together and share ways of
improving teaching\tabularnewline
Teacher Evaluation (De Facto) & Teacher Interview & Teacher Evaluation
(De Facto)\tabularnewline
Teacher reported being formally evaluated & Teacher Interview & Teacher
reported being formally evaluated\tabularnewline
Evaluation had multiple components & Teacher Interview & Evaluation had
multiple components\tabularnewline
Teacher reported negative consequences would occur if received 2 or more
negative evaluations & Teacher Interview & Teacher reported negative
consequences would occur if received 2 or more negative
evaluations\tabularnewline
Teacher reported positive consequences would occur if received 2 or more
positive evaluations & Teacher Interview & Teacher reported positive
consequences would occur if received 2 or more positive
evaluations\tabularnewline
Teacher Monitoring \& Accountability (De Facto) & Teacher Interview &
Teacher Monitoring \& Accountability (De Facto)\tabularnewline
Teachers evaluated by some authority on basis of absence & Teacher
Interview & Teachers evaluated by some authority on basis of
absence\tabularnewline
Teacher reported good attendance is rewarded. & Teacher Interview &
Teacher reported good attendance is rewarded.\tabularnewline
Teacher reported there are consequences for chronic absence (more than
30\% absence) & Teacher Interview & Teacher reported there are
consequences for chronic absence (more than 30\% absence)\tabularnewline
Teacher reported having to miss class for administrative reasons &
Teacher Interview & Teacher reported having to miss class for
administrative reasons\tabularnewline
Inputs and Infrastructure Standards & Principal Interview & Inputs and
Infrastructure Standards\tabularnewline
Inputs and Infrastructure Monitoring & Principal Interview & Inputs and
Infrastructure Monitoring\tabularnewline
Fraction of inputs Principal reported being monitored (functioning
blackboard, chalk, pens, pencils, textbooks, exercise books in 4th grade
classrooms, basic classroom furniture, and at least one computer in the
schools) & Principal Interview & Fraction of inputs Principal reported
being monitored (functioning blackboard, chalk, pens, pencils,
textbooks, exercise books in 4th grade classrooms, basic classroom
furniture, and at least one computer in the schools)\tabularnewline
Fraction of infrastructure elements Principal reported being monitored
(functioning toilets, electricity, drinking water, and accessibility for
people with disabilities) & Principal Interview & Fraction of
infrastructure elements Principal reported being monitored (functioning
toilets, electricity, drinking water, and accessibility for people with
disabilities)\tabularnewline
Principal reports community involved in the monitoring & Principal
Interview & Principal reports community involved in the
monitoring\tabularnewline
School Management Clarity of Functions & Principal Interview & School
Management Clarity of Functions\tabularnewline
Do you know if the policies governing schools assign responsibility for
the implementation of each of the following: Maintenance of
Infrastructure & Principal Interview & Do you know if the policies
governing schools assign responsibility for the implementation of each
of the following: Maintenance of Infrastructure\tabularnewline
Do you know if the policies governing schools assign responsibility for
the implementation of each of the following: Procurement of materials &
Principal Interview & Do you know if the policies governing schools
assign responsibility for the implementation of each of the following:
Procurement of materials\tabularnewline
Do you know if the policies governing schools assign responsibility for
the implementation of each of the following: Teacher hiring and
assignment & Principal Interview & Do you know if the policies governing
schools assign responsibility for the implementation of each of the
following: Teacher hiring and assignment\tabularnewline
Do you know if the policies governing schools assign responsibility for
the implementation of each of the following: Teacher supervision,
training, and coaching & Principal Interview & Do you know if the
policies governing schools assign responsibility for the implementation
of each of the following: Teacher supervision, training, and
coaching\tabularnewline
Do you know if the policies governing schools assign responsibility for
the implementation of each of the following: Student Learning
Assessments & Principal Interview & Do you know if the policies
governing schools assign responsibility for the implementation of each
of the following: Student Learning Assessments\tabularnewline
Do you know if the policies governing schools assign responsibility for
the implementation of each of the following: Principal hiring and
assignment & Principal Interview & Do you know if the policies governing
schools assign responsibility for the implementation of each of the
following: Principal hiring and assignment\tabularnewline
Do you know if the policies governing schools assign responsibility for
the implementation of each of the following: Principal Supervision and
training & Principal Interview & Do you know if the policies governing
schools assign responsibility for the implementation of each of the
following: Principal Supervision and training\tabularnewline
School Management Attraction & Principal Interview & School Management
Attraction\tabularnewline
Principal satisfied with status in community & Principal Interview &
Principal satisfied with status in community\tabularnewline
Principal reported salary as fraction of GDP & Principal Interview &
Principal reported salary as fraction of GDP\tabularnewline
School Management Selection \& Deployment & Principal Interview & School
Management Selection \& Deployment\tabularnewline
School Management Support & Principal Interview & School Management
Support\tabularnewline
Principal reported receiving formal training on managing school &
Principal Interview & Principal reported receiving formal training on
managing school\tabularnewline
Principal had received the following training (1/3 point for each):
Management training for new principals, in-service training, mentoring
coaching & Principal Interview & Principal had received the following
training (1/3 point for each): Management training for new principals,
in-service training, mentoring coaching\tabularnewline
Principal reported using skills from training & Principal Interview &
Principal reported using skills from training\tabularnewline
Principal reported offered training at least once per year & Principal
Interview & Principal reported offered training at least once per
year\tabularnewline
School Management Evaluation & Principal Interview & School Management
Evaluation\tabularnewline
Principal reported being formally evaluated & Principal Interview &
Principal reported being formally evaluated\tabularnewline
Principal evaluation based on multiple components (at least 5 components
for full credit & Principal Interview & Principal evaluation based on
multiple components (at least 5 components for full
credit\tabularnewline
Principal reported negative consequences would occur if received 2 or
more negative evaluations & Principal Interview & Principal reported
negative consequences would occur if received 2 or more negative
evaluations\tabularnewline
Principal reported negative consequences would occur if received 2 or
more negative evaluations & Principal Interview & Principal reported
negative consequences would occur if received 2 or more negative
evaluations\tabularnewline
National Learning Goals & Public Official Interview & Average of
Targeting, Monitoring, Incentives, and Community Engagement
components\tabularnewline
Targeting & Public Official Interview & Questions on whether learning
goals exist, are measurable, whether tasks aligned\tabularnewline
Monitorinig & Public Official Interview & Questions on how well
performance goals tracked, EMIS system, how info reviewed\tabularnewline
Incentives & Public Official Interview & Are rewards given for high
performance, rewards for excellence in contributing to learning goals,
does information inform budgets\tabularnewline
Community Engagement & Public Official Interview & Most common means of
getting feedback from parents, is feedback used for
evaluation/budgets\tabularnewline
Mandates and Accountability & Public Official Interview & Average of
coherence, transparency, accountability of public
officials\tabularnewline
Coherence & Public Official Interview & Is organizational
responsibilities clear\tabularnewline
Transparency & Public Official Interview & Are achievements related to
performance targets made public, is transparency
beneficial\tabularnewline
Accountability of Public Officials & Public Official Interview & What
would happen to public officials in cases of malfeasance\tabularnewline
Quality of Bureaucracy & Public Official Interview & Average score of
knowledge and skills, work environment, merit, movitation and
attitudes\tabularnewline
Knowledge and Skills & Public Official Interview & Public officials
asked if they know numbers on average class size, teacher
absences\tabularnewline
Work Environment & Public Official Interview & Do employees trust one
another, gifts to public officials, encouragement of new
ideas\tabularnewline
Merit & Public Official Interview & How is selection and deployment
done\tabularnewline
Motivation and Attitudes & Public Official Interview & How satisfied are
public officials, how motivated, is it OK for teachers to be absent, ok
for some students get less attention\tabularnewline
Motivation Relative to Starting Public Service & Public Official
Interview & Motivation Relative to Starting Public
Service\tabularnewline
Impartial Decision Making & Public Official Interview & Impartial
Decision Making\tabularnewline
Politicized personnel management & Public Official Interview & Are
hiring/promotion decisions/evaluations based on public
influence\tabularnewline
Politicized policy-making & Public Official Interview & Are
budget,procurement, curriculum decisions affected by
politics?\tabularnewline
Politicized policy-implementation & Public Official Interview & What
proportion of public officials broken rules, are contracts/procuements
subject to politics\tabularnewline
Employee unions as facilitators & Public Official Interview & Does union
membership affect teacher hiring, do public official union members get
preferential treatment, are new practices influence by
unions?\tabularnewline
\bottomrule
\end{longtable}

We will show a working example of how each indicator is scored in the
case of Peru.

\hypertarget{teacher-absence}{%
\subsubsection{Teacher Absence}\label{teacher-absence}}

Teacher absence is defined as whether or not the teacher was absenct
from the school or classroom during an unannounced visit. For a working
example, see below. The basis for the question is the following below:

What was the teacher doing when you located him/ her on the first
visit?\\
1= In classroom- teaching\\
2= In classroom- not teaching\\
3= At school- teaching outdoors\\
4= At school not his/her shift/not her class\\
5= At school- not in classroom\\
6= Absent from school

\begin{Shaded}
\begin{Highlighting}[]
\NormalTok{teacher_absence_ex<-teacher_roster_anon }\OperatorTok
\StringTok{  }\KeywordTok{select}\NormalTok{(teacher_number, m2sbq6_efft, teacher_available, m2sbq3_efft) }\OperatorTok
\StringTok{  }\KeywordTok{filter}\NormalTok{(}\OperatorTok{!}\KeywordTok{is.na}\NormalTok{(m2sbq6_efft) }\OperatorTok{|}\StringTok{ }\OperatorTok{!}\KeywordTok{is.na}\NormalTok{(m2sbq3_efft) )}


\CommentTok{#create indicator for whether each teacher was absent from school}
\NormalTok{teacher_absence_ex <-}\StringTok{ }\NormalTok{teacher_absence_ex }\OperatorTok
\StringTok{  }\KeywordTok{mutate}\NormalTok{(}\DataTypeTok{sch_absence_rate=}\DecValTok{100}\OperatorTok{*}\KeywordTok{case_when}\NormalTok{(}
\NormalTok{    m2sbq6_efft}\OperatorTok{==}\DecValTok{6} \OperatorTok{|}\StringTok{ }\NormalTok{teacher_available}\OperatorTok{==}\DecValTok{2} \OperatorTok{~}\StringTok{ }\DecValTok{1}\NormalTok{,}
\NormalTok{    m2sbq6_efft}\OperatorTok{!=}\DecValTok{6}   \OperatorTok{~}\StringTok{ }\DecValTok{0}\NormalTok{,}
    \KeywordTok{is.na}\NormalTok{(m2sbq6_efft) }\OperatorTok{~}\StringTok{ }\KeywordTok{as.numeric}\NormalTok{(}\OtherTok{NA}\NormalTok{)))}

\CommentTok{#create indicator for whether each teacher was absent from classroom or school}
\NormalTok{teacher_absence_ex <-}\StringTok{ }\NormalTok{teacher_absence_ex }\OperatorTok
\StringTok{  }\KeywordTok{mutate}\NormalTok{(}\DataTypeTok{absence_rate=}\DecValTok{100}\OperatorTok{*}\KeywordTok{case_when}\NormalTok{(}
\NormalTok{    m2sbq6_efft}\OperatorTok{==}\DecValTok{6} \OperatorTok{|}\StringTok{ }\NormalTok{m2sbq6_efft}\OperatorTok{==}\DecValTok{5} \OperatorTok{|}\StringTok{  }\NormalTok{teacher_available}\OperatorTok{==}\DecValTok{2} \OperatorTok{~}\StringTok{ }\DecValTok{1}\NormalTok{,}
\NormalTok{    m2sbq6_efft}\OperatorTok{==}\DecValTok{1} \OperatorTok{|}\StringTok{ }\NormalTok{m2sbq6_efft}\OperatorTok{==}\DecValTok{3} \OperatorTok{|}\StringTok{ }\NormalTok{m2sbq6_efft}\OperatorTok{==}\DecValTok{2} \OperatorTok{|}\StringTok{ }\NormalTok{m2sbq6_efft}\OperatorTok{==}\DecValTok{4}  \OperatorTok{~}\StringTok{ }\DecValTok{0}\NormalTok{,}
    \KeywordTok{is.na}\NormalTok{(m2sbq6_efft) }\OperatorTok{~}\StringTok{ }\KeywordTok{as.numeric}\NormalTok{(}\OtherTok{NA}\NormalTok{)) )}

\CommentTok{#create indicator for whether each principal was absent from school}
\NormalTok{teacher_absence_ex <-}\StringTok{ }\NormalTok{teacher_absence_ex }\OperatorTok
\StringTok{  }\KeywordTok{mutate}\NormalTok{(}\DataTypeTok{principal_absence=}\DecValTok{100}\OperatorTok{*}\KeywordTok{case_when}\NormalTok{(}
\NormalTok{    m2sbq3_efft}\OperatorTok{==}\DecValTok{8}  \OperatorTok{~}\StringTok{ }\DecValTok{1}\NormalTok{,}
\NormalTok{    m2sbq3_efft}\OperatorTok{!=}\DecValTok{8}   \OperatorTok{~}\StringTok{ }\DecValTok{0}\NormalTok{,}
    \KeywordTok{is.na}\NormalTok{(m2sbq3_efft) }\OperatorTok{~}\StringTok{ }\KeywordTok{as.numeric}\NormalTok{(}\OtherTok{NA}\NormalTok{))) }\OperatorTok
\StringTok{  }\KeywordTok{mutate}\NormalTok{(}\DataTypeTok{absence_rate=}\KeywordTok{if_else}\NormalTok{(}\KeywordTok{is.na}\NormalTok{(absence_rate), principal_absence, absence_rate ),}
         \DataTypeTok{sch_absence_rate=}\KeywordTok{if_else}\NormalTok{(}\KeywordTok{is.na}\NormalTok{(sch_absence_rate), principal_absence, sch_absence_rate ),}
         \DataTypeTok{presence_rate=}\DecValTok{100}\OperatorTok{-}\NormalTok{absence_rate)}

  \CommentTok{#produce caption for table}
\NormalTok{absence_captioner<-}\KeywordTok{tab_num}\NormalTok{(}\DataTypeTok{name=}\StringTok{"absence_cap"}\NormalTok{, }\DataTypeTok{caption=}\StringTok{"Data Example of How Absence Scored"}\NormalTok{)}
\KeywordTok{kable}\NormalTok{(}\KeywordTok{head}\NormalTok{(teacher_absence_ex, }\DataTypeTok{n=}\DecValTok{5}\NormalTok{),}
      \DataTypeTok{caption=}\NormalTok{absence_captioner)}
\end{Highlighting}
\end{Shaded}

\begin{longtable}[]{@{}cccccccc@{}}
\toprule
teacher\_number & m2sbq6\_efft & teacher\_available & m2sbq3\_efft &
sch\_absence\_rate & absence\_rate & principal\_absence &
presence\_rate\tabularnewline
\midrule
\endhead
1 & 6 & 1 & NA & 100 & 100 & NA & 0\tabularnewline
2 & 1 & 1 & NA & 0 & 0 & NA & 100\tabularnewline
3 & 1 & 1 & NA & 0 & 0 & NA & 100\tabularnewline
4 & 1 & 1 & NA & 0 & 0 & NA & 100\tabularnewline
5 & 1 & 1 & NA & 0 & 0 & NA & 100\tabularnewline
\bottomrule
\end{longtable}

\begin{Shaded}
\begin{Highlighting}[]
\KeywordTok{label}\NormalTok{(teacher_absence_ex}\OperatorTok{$}\NormalTok{m2sbq3_efft)}
\end{Highlighting}
\end{Shaded}

\begin{verbatim}
## [1] "What was the Principal doing when you located him/ her on the visit?"
\end{verbatim}

\begin{Shaded}
\begin{Highlighting}[]
\KeywordTok{label}\NormalTok{(teacher_absence_ex}\OperatorTok{$}\NormalTok{m2sbq6_efft)}
\end{Highlighting}
\end{Shaded}

\begin{verbatim}
## [1] "What was the teacher doing when you located him/ her on the visit?"
\end{verbatim}

\begin{Shaded}
\begin{Highlighting}[]
\KeywordTok{label}\NormalTok{(teacher_absence_ex}\OperatorTok{$}\NormalTok{teacher_available)}
\end{Highlighting}
\end{Shaded}

\begin{verbatim}
## [1] "Is the teacher available for interview"
\end{verbatim}

\hypertarget{student-attendance}{%
\subsubsection{Student Attendance}\label{student-attendance}}

Student attendance is based on the classroom observation of a randomly
selected fourth grade class. The teacher is asked for the class roster
and the enumerator then takes a count of the students in the classroom.
A working example is below.

\begin{Shaded}
\begin{Highlighting}[]
\CommentTok{#############################################}
\CommentTok{##### Student Attendance ###########}
\CommentTok{#############################################}

\CommentTok{#Percent of 4th grade students who are present during an unannounced visit.}

\NormalTok{school_data_INPT_ex<-}\StringTok{ }\NormalTok{school_dta_anon }\OperatorTok
\StringTok{  }\KeywordTok{select}\NormalTok{(m4scq4_inpt, m4scq12_inpt )  }\OperatorTok
\StringTok{  }\KeywordTok{filter}\NormalTok{(}\OperatorTok{!}\KeywordTok{is.na}\NormalTok{(m4scq4_inpt) }\OperatorTok{&}\StringTok{ }\OperatorTok{!}\KeywordTok{is.na}\NormalTok{(m4scq12_inpt)) }\OperatorTok
\StringTok{  }\KeywordTok{mutate}\NormalTok{(}\DataTypeTok{student_attendance=}\NormalTok{m4scq4_inpt}\OperatorTok{/}\NormalTok{m4scq12_inpt) }\OperatorTok
\StringTok{  }\KeywordTok{mutate}\NormalTok{(}\DataTypeTok{student_attendance=}\DecValTok{100}\OperatorTok{*}\NormalTok{student_attendance) }
  
\CommentTok{#produce caption for table}
\NormalTok{attendance_cap<-}\KeywordTok{tab_num}\NormalTok{(}\DataTypeTok{name=}\StringTok{"attendance_cap"}\NormalTok{, }\DataTypeTok{caption=}\StringTok{"Data Example of How Student Attendance Scored"}\NormalTok{)}
\KeywordTok{kable}\NormalTok{(}\KeywordTok{head}\NormalTok{(school_data_INPT_ex, }\DataTypeTok{n=}\DecValTok{5}\NormalTok{),}
      \DataTypeTok{caption=}\NormalTok{attendance_cap)}
\end{Highlighting}
\end{Shaded}

\begin{longtable}[]{@{}ccc@{}}
\toprule
m4scq4\_inpt & m4scq12\_inpt & student\_attendance\tabularnewline
\midrule
\endhead
20 & 21 & 95.23810\tabularnewline
30 & 33 & 90.90909\tabularnewline
18 & 20 & 90.00000\tabularnewline
28 & 28 & 100.00000\tabularnewline
10 & 10 & 100.00000\tabularnewline
\bottomrule
\end{longtable}

\begin{Shaded}
\begin{Highlighting}[]
\KeywordTok{label}\NormalTok{(school_data_INPT_ex}\OperatorTok{$}\NormalTok{m4scq4_inpt)}
\end{Highlighting}
\end{Shaded}

\begin{verbatim}
## [1] "How many pupils are in the room?"
\end{verbatim}

\begin{Shaded}
\begin{Highlighting}[]
\KeywordTok{label}\NormalTok{(school_data_INPT_ex}\OperatorTok{$}\NormalTok{m4scq12_inpt)}
\end{Highlighting}
\end{Shaded}

\begin{verbatim}
## [1] "How many students are in the class, according to the class list?"
\end{verbatim}

\hypertarget{teacher-knowledge}{%
\subsubsection{Teacher Knowledge}\label{teacher-knowledge}}

The teacher content knowledge score is based on a teacher assessment
given to teachers. The assessment is described in more detail above.
Math teachers were given a mathematics exam and Language teachers were
given a separate language exam. In cases where a teacher taught both
subjects, the teacher is randomly given either the math or language
exam. The knowledge score is the sum of the items answered correctly out
of the total number of items, based on either the math or language
assessment as is given. A detailed scoring example is omitted, because
of space, but details on scoring are provided in the GEPD github repo:

\url{https://github.com/worldbank/GEPD}

\hypertarget{th-grade-assessment}{%
\subsubsection{4th Grade Assessment}\label{th-grade-assessment}}

Students in a randomly chosen class were given an assessment containing
math and language items based on the SDI examination. More details on
this examination is above. A students score is the sum of the literacy
sub-score and the math sub-score. The literacy score is constructed so
that each literacy item is given the same weight and is the sum of
individual literacy items. Math scores are calculated similarly. The
final content knowledge score is then the sum of the two components. A
detailed scoring example is omitted, because of space, but details on
scoring are provided in the GEPD github repo:

Proficiency is then based on whether the student correctly answered a
set number of questions (12/13 in literacy, 14/17 in math), which was
determined by consulting a set of experts on whether a minimally
proficient 4th grader should be able to answer the questions correctly.

\hypertarget{ece-assessment}{%
\subsubsection{ECE Assessment}\label{ece-assessment}}

A randomly selected set of three 1st grade students are given an
assessment of early literacy, numeracy, executive functioning and
socio-emotional skills. The assessment was created based on items from
the MELQO assessment. A students score is the sum of the literacy
sub-score, the math sub-score, the executive functioning score, and the
socio-emotional score. Each domain (literacy, numeracy, executive
functioning and socio-emotional skills) is given equal weight. Within
each domain scores are based on the number of items correct in each
domain.

\hypertarget{psychometric-properties-of-4th-grade-assessment}{%
\subsection{Psychometric Properties of 4th Grade
Assessment}\label{psychometric-properties-of-4th-grade-assessment}}

For a full report on the psychometric properties of our assessment in
Peru, see:

\url{https://worldbankgroup-my.sharepoint.com/:u:/g/personal/bstacy_worldbank_org/EeI0nmjm-MlLl3OyPAfOzdsB0VBT7v-Q1JHrwig3H2FMEw?e=Pq2Fzg}

\hypertarget{psychometric-properties-of-1st-grade-assessment}{%
\subsection{Psychometric Properties of 1st Grade
Assessment}\label{psychometric-properties-of-1st-grade-assessment}}

For a full report on the psychometric properties of our assessment in
Peru, see:

\url{https://worldbankgroup-my.sharepoint.com/:u:/g/personal/bstacy_worldbank_org/EcwUNOUuTE1Ltjow7o10AUQBulnO_kKRPR0Oh8yTv5l7MA?e=miiiZU}

\hypertarget{psychometric-properties-of-teacher-assessment}{%
\subsection{Psychometric Properties of Teacher
Assessment}\label{psychometric-properties-of-teacher-assessment}}

For a full report on the psychometric properties of our assessment in
Peru, see:

\url{https://worldbankgroup-my.sharepoint.com/:u:/g/personal/bstacy_worldbank_org/EbHd7iWaxGhJspphCXLqkfwBAlzQ0hUjtUb8uT151CxL-A?e=82bgKE}


\end{document}
